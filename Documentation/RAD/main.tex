%%%%%%%%%%
% LaTeX Problem Set Template
% CISC 203: Discrete Mathematics for Computing II
% Queen's University, Fall 2020
% 
% Adapted for CISC 320 Project Proposal
%%%%%%%%%%

\documentclass{article}

%%%%%%%%%%
% YOUR INFO
%%%%%%%%%%

\newcommand{\MyName}{CleanShare}
\newcommand{\PSNumber}{1}

%%%%%%%%%%
% Packages
%%%%%%%%%%
\usepackage{amsmath, amssymb, amsthm, float}
\usepackage[margin=1in]{geometry}
\usepackage{fancyhdr}
\pagestyle{fancy}
\fancyhead[L]{\MyName}
\fancyhead[C]{CISC 320 Requirements Analysis Document}
\fancyhead[R]{Page \thepage}
\fancyfoot{}
\fancypagestyle{firstpage}{%
    \fancyhf{}
}
\usepackage{enumitem}
\usepackage{graphicx}
\usepackage{xcolor}
\usepackage{tikz, url}
\usetikzlibrary{arrows.meta, calc, chains, decorations.pathreplacing, graphs, graphs.standard, matrix, positioning, shapes, trees}

\usepackage{tocloft}
\setlength{\cftbeforesecskip}{1pt} % default is ~10pt
\setlength{\cftbeforesubsecskip}{0.5pt}

%%%%%%%%%%
% Metadata
%%%%%%%%%%
\title{CISC 320 Requirements Analysis Document
\\CleanShare}
\author{Section 3 - Group 6\\David Balan\\Anthony Grecu\\Charlie Guarasci\\Liam Harper-Mccabe\\Kieron Luke\\Mark Nistor\\Kayetan Protas}
\date{\today}

%%%%%%%%%%
% Document
%%%%%%%%%%
\begin{document}
\maketitle
\tableofcontents
\newpage



\section{Executive Summary}
In today’s digital age, maintaining a professional and polished social media presence is essential for our personal image, career opportunities, and overall reputation—especially with the rise of cancel culture and consequences for poor choices made online. So, meet CleanShare: an innovative web application designed to help users share photos that include certain adult beverages they may not want online.  

CleanShare allows users to simply drag and drop their .jpg or .png images into CleanShare's UI, where, with a click of a button, any alcoholic beverages will be automatically blurred. Whether you're a Queen’s student worried about maintaining a polished image for prospective employers or a professor wanting to keep their social life private, CleanShare allows you to share photos online without the worry of blemishing your image. 

\section{Background and History}
Universities maintain strict guidelines around public representation, especially concerning alcohol in the media. At Queen’s University, many student clubs and organizations promote events through social media. However, the presence of visible alcoholic beverages in photos can violate student conduct policies and damage reputations. Currently, such images must be manually edited—a process that is slow, inconsistent, and dependent on human judgment. 
 
CleanShare addresses this problem by providing a locally executed desktop application that detects and blurs alcoholic beverages in photos automatically. The tool combines traditional computer vision with a compact convolutional neural network (CNN) optimized for on-device inference. Detection is based on the Liquor Data dataset from Lamar University (Roboflow: \url{https://universe.roboflow.com/lamar-university-venef/liquor-data}), which contains labeled images of alcoholic containers (beer, wine, and spirits) suitable for object detection model training. By running fully offline, CleanShare ensures privacy and reliability while maintaining compatibility with university IT policies that restrict cloud-based content processing. 
 
The project prioritizes transparent, explainable functionality over black-box automation. Detection performance, model reliability, and processing latency are treated as measurable system attributes rather than secondary effects of implementation. The design goal is not aesthetic enhancement but compliance automation—reducing manual editing without compromising privacy or accuracy. 

\section{Purpose and Scope}
The purpose of CleanShare is to develop an efficient, privacy-respecting desktop tool that automatically detects and blurs alcoholic beverages in digital photographs before sharing. It enables students, clubs, and campus organizations to prepare “safe-to-share” media without manual redaction or third-party upload, aligning with institutional image and conduct policies. 

\subsection{System Scope}
Automatic detection and blurring: Alcoholic containers are identified using a hybrid detection pipeline combining OpenCV contour and color feature extraction with CNN-based classification via ONNX Runtime. 

Local inference: All computation runs on the user’s machine using pre-trained ONNX weights exported from a YOLOv8-based model fine-tuned on the Liquor Data dataset. 

Platform target: Windows 10+ desktop systems with C++17 or later, Qt 6.x, and OpenCV 4.x. 

Hardware constraints: Minimum 8 GB RAM, CPU-only inference under 10 s per 1080p image. GPU acceleration (CUDA) optional. 

\subsection{Assumptions and Dependencies}
Functional Qt and OpenCV libraries installed in the runtime environment. Users provide static image files; video or batch modes are out of scope for the current release. 

\subsection{Validation and Evaluation}
System evaluation uses a held-out subset of Liquor Data annotated images. Performance metrics include: Precision/Recall and F1 score to measure detection correctness. Mean Average Precision (mAP@0.5) as the principal accuracy indicator. False positive rate against non-alcoholic container images (e.g., soda cans). A successful implementation must achieve $\geq$ 80\% recall, $\leq$ 15\% false positives, and average processing time $\leq$ 10 s for 1080p input. 

\subsection{Stakeholders}
Primary: University students, club social media coordinators, campus event organizers.  
Secondary: Faculty communications staff, student services offices, and event photographers. 

\section{Objectives and Success Criteria}
The objective of CleanShare is to provide a privacy-preserving desktop application that automatically detects and blurs alcoholic beverages in images before sharing. The system will allow users, particularly students and campus organizations, to safely post photos online without revealing intoxicating beverages that may conflict with institutional standards or professional expectations. The software must run entirely on the user’s computer, ensuring no compromising images or personal data leave the device. 
 
The minimum viable product will enable users to load an image, automatically detect and blur alcoholic beverages using a pre-trained model, view a preview of the processed image, make optional manual adjustments to blur regions, and export the final blurred image at its original resolution. The interface should support basic user workflow from import to export with no network connection required. The application will use Qt for the interface, OpenCV for preprocessing and redaction, and ONNX Runtime for model inference. 
 
Success will be measured by detection accuracy and system performance. The detection model must achieve at least 80 percent recall when evaluated on a held-out subset of labeled test images, while also maintaining a low false positive rate. The application must process a 1080p image in 10 seconds or less on typical laptop hardware without GPU acceleration. All processing will occur locally, and no user images will be transmitted or stored beyond temporary in-memory data. Manual editing tools must allow users to correct missed detections before exporting. The exported images must match the original resolution and contain all selected blur regions. If no alcohol is detected, the system will notify the user and still allow manual marking and export. Once these core requirements are achieved, additional improvements such as pixelation redaction options, performance optimizations, or batch processing may be considered for future releases. 

\section{Definitions and Acronyms}
CleanShare is a Windows desktop application that automatically detects and blurs alcoholic beverages in photos so they’re safe to share. It processes images locally (no cloud), using a C++/Qt GUI, OpenCV for image handling, and a bundled CNN served via ONNX Runtime for robust detection. Core user features include image upload, automatic detection+blur, side-by-side preview, manual correction tools, and export at the original resolution. Target performance is $\leq$ 10 s per 1080p photo with $\geq$ 80\% recall and low false positives. 

\subsection{Primary modules (logical view)}
\begin{itemize}
\item Presentation (Qt GUI): File open/save, side-by-side preview, blur-strength controls, and manual tools (brush/rectangle) to add/remove regions.
\item Application Core: Orchestrates the workflow (import → detect → blur → preview → export), holds session state and configuration (e.g., detection threshold, blur type).
\item Detection Engine: Image pre-processing → CNN inference (ONNX Runtime) → post-processing (NMS, thresholding). Packaged, pre-trained CNN weights enable local inference. A simple OpenCV heuristic may exist as a baseline/fallback.
\item Redaction Pipeline: Builds ROI masks from detections and applies blurring while preserving original dimensions on export; future variants (pixelation/mosaic) are optional extensions.
\item I/O \& Privacy: Supports JPEG/PNG input and exports blurred images; only ephemeral metadata (boxes/masks) is held in memory; no images or personal data leave the device.
\item Evaluation (dev utility): Dataset-based checks for recall/false positives and timing against the success criteria.
\end{itemize}

\section{General Description of the System}
Primary use case: “Blur a photo and export” 
\begin{enumerate}
\item Launch \& Idle: App starts → main window (no network required). Platform target: Windows.
\item Import: User selects an image (JPEG/PNG). App validates format, loads pixels, records original dimensions.
\item Pre-process for Detection: Convert to detector input (resize/normalize); keep a mapping to original coordinates.
\item Detect Alcohol Containers: Run CNN (ONNX Runtime). Optionally fuse heuristic detection; perform NMS and confidence thresholding.
\item Post-process Detections: Map boxes/masks back; build ROI mask.
\item Apply Redaction: Apply blur (default Gaussian) to ROI mask.
\item Preview \& Manual Adjustments: Show side-by-side; user may edit blur regions.
\item Export: Write blurred image; clear metadata.
\end{enumerate}

\subsection{Alternate flows \& error handling}
\begin{itemize}
\item No detections: Notify “No alcohol detected.” Allow manual marking.
\item Invalid file: Show an error and return to Import.
\item Slow inference: Show progress and allow cancel; fallback heuristic.
\item Privacy: No images or user data are sent to any server.
\end{itemize}

\subsection{(Optional) Batch flow (future extension)}
User selects a folder → app iterates the pipeline per image, applies consistent settings, and writes outputs to an export folder. 

\section{Program Flow}
\subsection{Program Flow Diagram}
The program flow diagram illustrates the overall execution sequence of CleanShare — from user input to image export. It shows how user actions (such as importing an image) interact with system processes like preprocessing, detection, blurring, and export. This helps visualize the step-by-step control flow and major function calls within the application.

\begin{figure}[H]
    \centering
    \includegraphics[width=0.85\textwidth]{Acitivty.png} % replace with your actual file name
    \caption{Program Flow Diagram for CleanShare}
    \label{fig:program_flow}
\end{figure}

\section{Functional Requirements}
\begin{itemize}
\item Allows users to upload images in JPEG, PNG.
\item Verifies the file type and size before processing.
\item Detect and blur alcoholic beverages automatically using CNN (ONNX Runtime).
\item Apply non-maximum suppression (NMS) and confidence thresholding.
\item Identifies bounding boxes or regions and applies Gaussian blur.
\item Offer preview mode with side-by-side comparison.
\item Allow users to manually correct using brush/rectangle tools.
\item Allow exporting at original resolution.
\item Handle failed uploads and model inference errors gracefully.
\item Ensure all processing remains local with no data leaving the device.
\end{itemize}

\section{Non-Functional Requirements }
\subsection{Performance requirements}
\begin{itemize}
\item Process a 1080p image in $\leq$ 10 s.
\item CNN detection recall $\geq$ 80\%, low false positives.
\item GUI response time $\leq$300 ms.
\end{itemize}

\subsection{User experience}
\begin{itemize}
\item Workflow intuitive for first-time users.
\item Consistent layout, labeled controls.
\item All major operations in $\leq$ 3 user actions.
\item Visual feedback for operations.
\item Undo/redo supported without corruption.
\item High-contrast actionable buttons.
\end{itemize}

\subsection{Ethical}
\begin{itemize}
\item Evaluate model for bias.
\item Notify users about false positives.
\end{itemize}

\subsection{Documentation}
\begin{itemize}
\item User manual with installation, usage, troubleshooting.
\item Developer guide with architecture and build process.
\item CNN model and preprocessing pipeline versioned for reproducibility.
\end{itemize}

\section{Layout of GUI}
\subsection{Landing Page}
Serves as the user's entry point to the application. Features a central area for uploading images via drag-and-drop or browse. Displays core features: Auto Detection, Privacy First, and Manual Control.
\begin{figure}[H]
    \centering
    \includegraphics[width=0.9\textwidth]{Home Page.png} % replace with your image name
    \caption{CleanShare Home Page Interface}
    \label{fig:homepage}
\end{figure}

\subsection{Edit/View Page}
Loads after processing. Side-by-side comparison of before/after. Right sidebar provides export options, file type, insights, and privacy policy. Users can download or upload a new image directly.
\begin{figure}[H]
    \centering
    \includegraphics[width=0.9\textwidth]{Edit-View.png} % replace with your image name
    \caption{CleanShare Edit/View Interface}
    \label{fig:uml}
\end{figure}

\section{Assumptions and Constraints}
We assume all members attend meetings, communicate changes, and complete tasks on time. Users will use JPEG/PNG files as input.  

Constraints include limited timeframe restricting testing and development, learning new technologies like OpenCV, and the need to run offline without cloud APIs.

\section{Roles of the Team}

\begin{table}[H]
\centering
\renewcommand{\arraystretch}{1.3}
\begin{tabular}{|p{4cm}|p{4cm}|p{7cm}|}
\hline
\textbf{Team Member(s)} & \textbf{Role} & \textbf{Responsibilities} \\ \hline
Anthony Grecu \& David Balan & Backend Developers & Handle image processing and develop machine learning model. \\ \hline
Mark Nistor \& Liam Harper-Mccabe & System Engineers & Integrate the back end and front end systems. \\ \hline
Kieron Luke \& Charlie Guarasc & Front End Developers & Design and create the GUI. \\ \hline
Kayetan Protas & Documentation and Quality Assurance Engineer & Document the project and create test scripts. \\ \hline
\multicolumn{3}{|c|}{\textit{Every team member assumes a coding role.}} \\ \hline
\end{tabular}
\caption{Roles and Responsibilities of the CleanShare Team}
\label{tab:team_roles}
\end{table}

\section{System Diagrams}

\subsection{Logical View Diagram}
The logical view diagram presents the main architectural components of CleanShare and their relationships. It shows how modules such as the GUI, Application Core, Detection Engine, and Redaction Pipeline interact logically. This diagram provides a high-level understanding of the system’s structure and separation of responsibilities.

\begin{figure}[H]
    \centering
    \includegraphics[width=0.9\textwidth]{logical.png} % replace with your actual file name
    \caption{Logical View Diagram of CleanShare Architecture}
    \label{fig:logical_view}
\end{figure}




\end{document}
