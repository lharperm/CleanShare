%%%%%%%%%%
% LaTeX Problem Set Template
% CISC 203: Discrete Mathematics for Computing II
% Queen's University, Fall 2020
% 
% Adapted for CISC 320 Project Proposal
%%%%%%%%%%

\documentclass{article}

%%%%%%%%%%
% YOUR INFO
%%%%%%%%%%

\newcommand{\MyName}{CleanShare}
\newcommand{\PSNumber}{1}

%%%%%%%%%%
% Packages
%%%%%%%%%%
\usepackage{amsmath, amssymb, amsthm}
\usepackage[margin=1in]{geometry}
\usepackage{fancyhdr}
\pagestyle{fancy}
\fancyhead[L]{\MyName}
\fancyhead[C]{CISC 320 Project Outline}
\fancyhead[R]{Page \thepage}
\fancyfoot{}
\fancypagestyle{firstpage}{%
    \fancyhf{}
}
\usepackage{enumitem}
\usepackage{graphicx}
\usepackage{xcolor}
\usepackage{tikz, url}
\usetikzlibrary{arrows.meta, calc, chains, decorations.pathreplacing, graphs, graphs.standard, matrix, positioning, shapes, trees}

%%%%%%%%%%
% Metadata
%%%%%%%%%%
\title{CISC 320 Project Outline\\CleanShare}
\author{Section 3 - Group 6\\David Balan\\Anthony Grecu\\Charlie Guarasci\\Liam Harper-Mccabe\\Kieron Luke\\Mark Nistor\\Kayetan Protas}
\date{\today}

%%%%%%%%%%
% Document
%%%%%%%%%%
\begin{document}
\maketitle

\section*{General Purpose}
CleanShare is a desktop application designed to automatically detect and blur alcoholic beverages in photos before they are shared. The purpose is to make images safe for posting on university-affiliated social media or sharing with family without manual editing, ensuring compliance with student guidelines and community standards.

\section*{Intended Audience}
\begin{itemize}
    \item \textbf{Primary:} University students, club social media managers, and campus event organizers who need to share photos publicly while avoiding alcohol-related content.  
    \item \textbf{Secondary:} High-school counselors, event photographers, and anyone who needs ``safe-to-share'' photos in a professional or family context.  
\end{itemize}

\section*{Features}
\begin{itemize}
    \item \textbf{Image Upload \& Export:} Users can submit a single image (initially) and export the blurred result in the same dimensions as the original.  
    \item \textbf{Automatic Detection \& Blur:} The system uses OpenCV for object detection, enhanced with a CNN model for higher accuracy. Any detected alcohol container, whether by branding or shape, will be blurred.  
    \item \textbf{Manual Adjustment Tools:} Users can refine blur regions with basic brush or rectangle tools.  
    \item \textbf{Preview:} Side-by-side before/after comparison before saving.  
    \item \textbf{Batch Processing (Stretch Goal):} Start with single-image support; expand to batch mode later in the project.  
    \item \textbf{Privacy by Design:} All processing is done locally on the user’s machine so no cloud uploads.  
\end{itemize}

\section*{Data Handled}
\begin{itemize}
    \item \textbf{Input:} User-submitted photos in common formats (JPEG, PNG).  
    \item \textbf{Processing Data:} Temporary metadata (bounding boxes, masks, CNN predictions) generated during blur operations.  
    \item \textbf{Output:} Exported images with blurred alcohol content. No personal information is stored.  
\end{itemize}

\section*{Technical Approach}
\begin{itemize}
    \item \textbf{Platform Priority:} Windows-only for grading and development.  
    \item \textbf{Tech Stack:} C++ with Qt for the GUI, OpenCV for image handling and baseline detection, and ONNX Runtime to integrate a bundled CNN model for robust alcohol detection.  
    \item \textbf{Model:} Pre-trained CNN weights will be packaged with the repo for local inference.  
    \item \textbf{Dataset:} Team will use a custom image set of various alcohol bottles to test detection and accuracy.  
\end{itemize}

\section*{Relevance \& Impact}
CleanShare reduces the time and effort required for manual redaction of alcohol in photos. It provides a practical tool for students, clubs, and organizations to share photos responsibly, while ensuring compliance with university and community policies.

\section*{Motivation}
Many student organizations face restrictions when posting event photos that display alcohol. Manually blurring drinks is not only time-consuming but also inconsistent and prone to error, often leading to compliance risks or damaged reputations. In today’s digital-first environment, where student groups and individuals are judged by what they post online, the stakes are higher than ever. A single overlooked bottle in a photo could cause disciplinary action, loss of funding for clubs, or embarrassment in a professional or family setting.  

By automating this process, CleanShare addresses a problem that is both practical and reputational. It saves organizations countless hours of tedious editing while creating a uniform standard of “safe-to-share” imagery. Beyond convenience, it actively protects students, clubs, and institutions from unnecessary risks. This is not simply a matter of efficiency; it is a matter of digital responsibility, reputation management, and creating an environment where people can confidently share their experiences without fear of unintended consequences.  

\section*{Functional Requirements}
The functional requirements describe what CleanShare must provide for its users:
\begin{itemize}
    \item The system must allow users to upload images in standard formats (JPEG, PNG).  
    \item The system must detect and blur alcoholic beverages automatically using computer vision techniques.    
    \item The system must offer a preview mode with a side-by-side comparison before saving.  
    \item The system must allow exporting of the processed image in the same resolution as the original.  
    \item The system must process images locally without requiring internet access or cloud uploads.  
\end{itemize}

\section*{Non-Functional Requirements}
The non-functional requirements define the quality, performance, and constraints of the system:
\begin{itemize}
    \item \textbf{Performance:} A 1080p photo should be processed in under 10 seconds on a typical Windows laptop.  
    \item \textbf{Usability:} The workflow should be intuitive enough that a first-time user can blur and export an image in under 30 seconds.  
    \item \textbf{Reliability:} The system should consistently detect at least 80\% of alcohol containers while keeping false positives to a minimum.  
    \item \textbf{Security:} All processing occurs locally; no personal data or images leave the user’s machine.  
    \item \textbf{Portability:} The initial release will target Windows; later versions may support other platforms.  
    \item \textbf{Maintainability:} The system should be built with modular components (GUI, detection, processing) to simplify future improvements.  
\end{itemize}

\section*{Evaluation and Success Criteria}
We will evaluate CleanShare through a custom dataset of alcohol-containing images. Success will be measured by:
\begin{itemize}
    \item Achieving recall above 80\% (most alcoholic containers are detected).  
    \item Maintaining low false positives (non-alcoholic items rarely blurred).  
    \item Processing a 1080p photo in under 10 seconds on a typical Windows laptop.  
    \item Delivering an intuitive workflow where a user can blur and export in under 30 seconds.  
\end{itemize}

\section*{Milestones / Development Plan}
\begin{itemize}
    \item Week 1--2: Develop Qt GUI framework; implement manual blur tool.  
    \item Week 3: Add OpenCV-based heuristic detection; enable preview and save.  
    \item Week 4: Integrate CNN detection with ONNX Runtime.  
    \item Week 5: Testing, refinement, dataset evaluation, and documentation.  
\end{itemize}

\section*{Limitations and Risks}
\begin{itemize}
    \item False positives (e.g., soda cans blurred incorrectly).  
    \item False negatives in low-light or crowded images.  
    \item Dataset bias due to limited examples of certain alcohol types.  
    \item Project timeline may restrict advanced features like batch processing.  
\end{itemize}

\section*{Future Extensions}
\begin{itemize}
    \item Batch processing for entire image folders.  
    \item Multiple blur styles (Gaussian, pixelation, mosaic).  
    \item Social media integration for direct posting.  
    \item User-defined presets for sensitivity and blur strength.  
\end{itemize}

\section*{References}

\begin{enumerate}
    % Core Libraries
    \item Bradski, G. (2000). The OpenCV Library. \textit{Dr. Dobb's Journal of Software Tools}.
    \item Qt Company. (2024). \textit{Qt Documentation}. Retrieved from\\ \url{https://doc.qt.io}
    \item Microsoft. (2024). \textit{ONNX Runtime Documentation}. Retrieved from\\ \url{https://onnxruntime.ai}
    
    % Object Detection and CNNs
    \item Redmon, J., Divvala, S., Girshick, R., \& Farhadi, A. (2016). You Only Look Once: Unified, Real-Time Object Detection. \textit{Proceedings of the IEEE Conference on Computer Vision and Pattern Recognition (CVPR)}.
    \item He, K., Zhang, X., Ren, S., \& Sun, J. (2016). Deep Residual Learning for Image Recognition. \textit{Proceedings of the IEEE Conference on Computer Vision and Pattern Recognition (CVPR)}.
    \item Kingma, D. P., \& Ba, J. (2015). Adam: A Method for Stochastic Optimization. \textit{International Conference on Learning Representations (ICLR)}.
    
    % Image Processing Theory
    \item Gonzalez, R. C., \& Woods, R. E. (2018). \textit{Digital Image Processing} (4th ed.). Pearson.
    \item Szeliski, R. (2022). \textit{Computer Vision: Algorithms and Applications} (2nd ed.). Springer.  

    % Practical "How-To" Resources
    \item OpenCV Team. (2024). \textit{Image Blurring with OpenCV}. Retrieved from\\ \url{https://docs.opencv.org/4.x/d4/d13/tutorial_py_filtering.html}
    \item Qt Company. (2024). \textit{Qt Widgets Tutorial}. Retrieved from\\ \url{https://doc.qt.io/qt-6/widgets-tutorial.html}
    \item ONNX. (2024). \textit{Exporting Models to ONNX}. Retrieved from\\ \url{https://onnx.ai/getting-started.html}
    \item Hugging Face. (2024). \textit{Hugging Face — The AI community building the future}. Retrieved from\\ \url{https://huggingface.co/}
    \item GeeksforGeeks. (2025, May 30). \textit{Introduction to GUI Programming in C++}. Retrieved from\\ \url{https://www.geeksforgeeks.org/cpp/cpp-gui-programming/}
    \item Towards Data Science. (2024). \textit{Convolutional Neural Networks for Beginners}. Retrieved from \\ \url{https://towardsdatascience.com/convolutional-neural-networks-for-beginners-c1de55eee2b2/}

\end{enumerate}

\end{document}
